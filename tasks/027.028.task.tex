Буер, весящий вместе с пассажирами $Q=1962$ Н, движется прямолинейно по
гладкой горизонтальной поверхности льда вследствие давления ветра
на парус, плоскость которого $ab$ образует угол $45^{\circ}$ с
направлением движения. Абсолютная скорость $\omega$ ветра
перпендикулярна направлению движения. Величина силы давления ветра
$P$ выражается формулой Ньютона:$P=kSu^2\cos^2\phi$,
где $\phi$ --- угол, образуемый относительной скоростью ветра $u$ с
перпендикуляром $N$ к плоскости паруса, $S=5$ м$^2$ --- площадь
паруса, $k=0.113$ --- опытный коэффициент. Сила давления $P$ направлена
перпендикулярно плоскости $ab$. Пренебрегая трением, найти:
$1)$ какую наибольшую скорость $v_{max}$ может получить буер;
$2)$ какой угол $\alpha$ составляет при этой скорости помещенный на мачте
флюгер с плоскостью паруса;
$3)$ какой путь $x_{1}$ если его начальная скорость равна нулю.
