Цилиндр веса $P$, радиуса $r$ и высоты $h$ подвешен на пружине $AB$, верхний конец которой $B$ закреплен;
цилиндр погружен в воду. В положении равновесия цилиндр погружается в воду на половину своей высоты.
В начальный момент времени цилиндр был погружен в воду на $\frac{2}{3}$ своей высоты и затем без начальной
скорости пришел в движение по вертикальной прямой. Считая жесткость пружины равной $c$ и предполагая, 
что действие воды сводится к добавочной архимедовой силе, определить движение цилиндра относительно положения
равновесия. Принять удельный вес воды равным $\gamma$.
