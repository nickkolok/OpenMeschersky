Стержень $OA$ вращаеться вокруг оси $z$ проходящей через точку $O$, с угловым 
замедлением $10 rad/c^2$. Вдоль стержня от точки $O$ скользит шайба $M$.
Определить абсолютное ускорение шайбы в момент, когда она находиться на
расстоянии $0,6 m$ от точки $O$ и имеет скорость  и ускорение в движении 
вдоль стержня соответсвенно $1,2 m/c$ и $0,9 m/c^2$, усли в этот момент
угловая скорость стержень равна $5 rad/c$.
  Ответ: $\omega_{a} = 15,33 m/c^2$ и составляет с направлениями $MO$ 
  угол в $23^{\circ}$.