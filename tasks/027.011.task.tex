При скоростном спуске лыжник массы $90$ кг скользил по склону в
$45^{\circ}$, не отталкиваясь палками. Коэффициент трения лыж о снег
$f=0.1$. Сопротивление воздуха движению лыжника пропорционально квадрату
скорости лыжника и при скорости в $1$ м/с равно $0.635$ Н. Какую
наибольшую скорость мог развить лыжник? Насколько увеличится
максимальная скорость, если подобрав лучшую мазь, лыжник уменьшит
коэффициент трения до $0.005$?
