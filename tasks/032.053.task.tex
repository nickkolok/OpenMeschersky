Пластина $D$ массы $100 g$, повешенная на пружинине $AB$ в неподвижной точке
$A$, движеться между полюсами магнита. Вследствие вихревых токов движение 
тормозится силой, пропорциональной скорости. сила сопративления движению
равна $kv\phi^2 H$, где $ k=0,0001$, $v$- скорость в m/c, $\phi$-магнитный поток 
между полюсами $N$ и $S$. В начальный момент скорость пластинки равна
нулю и пружина не растянута. Удлинение ее на $1 m$ получается при статическом
действии силы в $19,6 H$, приложенной в точке $B$. определить движение 
пластинки в том случае, когда $\Phi=10\sqrt{5}$ Вб (вебер-еденица магнитного
потока в СИ).
