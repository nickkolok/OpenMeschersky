+(Б.Н. Садовский)
Если скорости двух точек, неподвижных относительно $\tilde{S}$ , в любой момент
времени одинаковы относительно $S$ , то можно ли утверждать, что матри-
ца $B$ перехода от координат векторов в подвижной системе $\tilde{S}$ к коорди-
натам векторов в неподвижной системе $S$ постоянна?
\vartrianglerightНет. Например, если 


  \begin{pmatrix}
    \cost $\-sint$ $0$
    \sint \cost $0$     ,$\tilde{r}_{\tilde{O}\tilde{A_{1}}$=\begin{pmatrix} ,    $\tilde{r}_{\tilde{O}\tilde{A_{2}}$=\begin{pmatrix} ,
    $0$ $0$ $1$                                                $0$                                                      $0$
   \end{pmatrix}                                               $0$                                                      $0$
                                                               $1$                                                      $2$
                                                                                                                       \end{pmatrix}
 то $v_{\barA_1}=v_{\bar O}+\dotB \tilde r_{\bar O \bar A_{1}}=v_{\bar O}=v_{\bar A_{2}$,но матрица $B$ не постоянна.\vartriangleft                                                                                            
                                                               
                                                               
                                                              
