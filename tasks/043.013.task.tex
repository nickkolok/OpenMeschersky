На боковой поверхности круглого цилиндра с вертикальной осью,
вокруг которой он может вращаться без трения,
вырезан гладкий винтовой жёлоб с углом подъёма $\alpha$.
В начальный момент цилиндр находится в покое; в жёлоб опускают тяжёлый шарик;
он падает по жёлобу без начальной скорости и заставляет цилиндр вращаться.
Дано: масса цилиндра $M$, радиус его $R$, масса шарика $m$;
расстояние от шарика до оси считаем равным $R$ и момент инерции цилиндра
$\frac{1}{2}MR^2$.
Определить угловую скорость $\omega$, которую целиндр будет иметь в тот момент,
когда шарик опустится на высоту $h$.
