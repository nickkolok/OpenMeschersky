К однородному стержню, длина которого $3$ м, а вес $6$ Н, подвешены $4$
груза на равных расстояниях друг от друга, причём два крайних --- на
концах стержня. Первый груз слева весит $2$ Н, каждый последующий
тяжелее предыдущего на $1$ Н. На каком расстоянии $x$ от левого конца
нужно подвесить стержень, чтобы он оставался горизонтальным?
