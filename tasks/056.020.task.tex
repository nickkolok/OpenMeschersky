Агрегат, состоящий из двигателя $1$ и машины $2$,
соединённых упругой муфтой $3$ с жёсткостью $c$,
рассматривается как двухмассовая система.
К ротору двигателя, имеющему момент инерции $J_1$,
приложен момент $M_1$, зависящий от угловой скорости ротора $\dot{\varphi}$:
$$M_1 = M_0 - \mu _1(\dot{\varphi} - \omega _0).$$
К валу машины, имеющему момент инерции $J_2$, приложен момент сил сопротивления,
зависящий от угловой скорости вала $\dot{\psi}$:
$$M_2 = M_0 - \mu _2(\dot{\psi} - \omega _0).$$
Коэффициенты $\mu _1$ и $\mu _2$ положительны.
Определить условия, при которых вращение системы с угловой скоростью
$\omega _0$ является устойчивым.
