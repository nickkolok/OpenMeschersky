Для поворота корпуса космического аппарата используется
электродвигатель-маховик, уравнение движения которого на вращающемся аппарате
имеет вид $\dot{\omega} + \omega / T = u$, где $\omega$ --- относительная
угловая скорость маховика, $T$ --- его постоянная времени, $u$ --- управляющее
напряжение, принимающее значение $\pm u_0$.
Определить длительность $t_1$ разгона ($u = u_0$)
и торможения $t_2$($u = -u_0$) маховика, если первоначально невращающийся
корпус при неподвижном маховике требуется повернуть
на заданный угол $\varphi$ и остановить.
Ось вращения маховика проходит через центр масс космического аппарата;
движение считать плоским.
Моменты инерции маховика и аппарата относительно общей оси вращения
соответственно равны $J$ и $J_0$.
