Подводная лодка, не имевшая хода, получив небольшую отрицательную
плавучесть $p$, погружается на глубину, двигаясь поступательно.
Сопротивление воды при небольшой отрицательной плавучести можно
принять пропроциональным первой степени скорости погружения и равным
$kSv$, где $k$ --- коэффициент пропорциональности, $S$ --- площадь
горизонтальной проекции лодки, $v$ --- величина скорости погружения.
Масса лодки равна $M$. Определить скорость погружения $v$, если при
$t=0$ скорость $v_{0}=0$.
