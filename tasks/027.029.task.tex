Вожатый трамвая, выключая постепенно реостат, увеливает
мощность вагонного двигателя так, что сила тяги возрастёт от нуля
пропорционально времени, увеличиваясь на $1200$ Н в течение каждой
секунды. Найти зависимость пройденного пути от времени движения вагона
при следующий данных: масса вагон $10 000$ кг, сопротивление трения
постоянно и составляет $0.02$ веса вагона, а начальная скорость
равна нулю.
