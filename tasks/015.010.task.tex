Кривошип $OA$ антипараллелограмма $OABO_1$,
поставленного на малое звено $OO_1$,
равномерно вращается с угловой скоростью $\omega$.
Приняв за полюс точку $A$, составить уравнения движения звена $AB$,
если $OA = O_1 B = a$ и $OO_1 = AB = b$ $(a > b)$;
в начальный момент кривошип $OA$ был направлен по $OO_1$.
