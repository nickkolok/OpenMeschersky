Шарик массы $m$ закреплен на конце вертикального упругого стержня,
зажатого нижним концом в неподвижной стойке.
При небольших отклонениях стержня от его вертикального
равновесного положения можно приближенно считать, что центр
шарика движется в горизонтальной плоскости $Oxy$, проходящей через
верхнее равновесное положение центра щарика.
Определить закон изменения силы, с которой упругий, изогнутый стержень
действует на шарик,если выведенный из своего положения равновесия, 
приятного за начало координат, шарика движется согласно уравнениями
$x=\alpha\cos kt$, $y=b\sin kt$, где $a$,$b$,$k$ --- постоянные величины
