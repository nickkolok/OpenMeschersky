$\omega = \sqrt{n^2_1 + n^2_2 + 2n_1n_2\cos{\theta _0}}$;
неподвижный аксоид --- круговой конус
$\xi ^2 + \eta ^2 -
\frac{b^2_2 \sin^2{\theta _0}}{(n_2\cos{\theta _0} + n_1)^2}\zeta ^2 = 0$
с осью $\zeta$ и углом раствора $2\arcsin{\frac{n_2\sin{\theta _0}}{\omega}}$;
подвижный аксоид --- круговой конус
$x^2 + y^2 - \frac{n^2_1 \sin{\theta _0}}{(n_1\cos{\theta _0} + n_2)^2}z^2 = 0$
с осью $z$ и углом раствора $2\arcsin{\frac{n_1\sin{\theta _0}}{\omega}}$.
