Три груза массы $M_{1}=20$ кг,$M_{2}=15$ кг и $M_{3}=10$ кг соединены нерастяжимой нитью, переброшенной через неподвижные блоки L и N. При опускании груза $M_{1}$ вниз груз $M_{2}$
перемещается по верхнему основанию четырехугольной усеченной пирамиды $ABCD$ массы $M=100$ кг вправо, а груз $M_{3}$ поднимается по боковой грани AB вверх. Пренебрегая трением между
усеченной пирамидой ABCD и полом, определить перемещение усеченной пирамиды $ABCD$ относительно пола, если груз $M_{1}$ опустится вниз на $1$ м. Массой нити пренебречь.
